% -*- coding: utf-8; -*-
\documentclass{article}

%% This helper document is part of QSoas, it's purpose is to make
%% the computations involved in some parts of the code more explicit
%% 
%% Copyright 2024 by CNRS/AMU

%% This program is free software; you can redistribute it and/or
%% modify it under the terms of the GNU General Public License as
%% published by the Free Software Foundation; either version 2 of the
%% License, or (at your option) any later version.

%% This program is distributed in the hope that it will be useful, but
%% WITHOUT ANY WARRANTY; without even the implied warranty of
%% MERCHANTABILITY or FITNESS FOR A PARTICULAR PURPOSE.  See the GNU
%% General Public License for more details.

%% You should have received a copy of the GNU General Public License
%% along with this program.  If not, see
%% <http://www.gnu.org/licenses/>.

\usepackage[english]{babel}
\usepackage[utf8]{inputenc}
\usepackage[T1]{fontenc}

\usepackage{amsmath}

\usepackage[svgnames]{xcolor}
\usepackage[colorlinks, urlcolor=blue, citecolor=DarkGreen,
linkcolor=blue]{hyperref}

\usepackage{graphicx}

\usepackage[a4paper,landscape=false,showframe=false,
margin=1.8cm]{geometry}

\usepackage[version=3]{mhchem}

\usepackage{siunitx}
\usepackage{booktabs}

\usepackage{mathpazo}
\usepackage[scaled]{helvet}
\usepackage{courier}

\usepackage{cite}
\usepackage{pifont}



%% A series of useful macros:
\newcommand{\dd}{\mathrm{d}}
\newcommand{\pa}[1]{\left(#1\right)}
\newcommand{\pab}[1]{\left[#1\right]}
\newcommand{\obs}[2]{\textbf{Note: #1 --} #2}


\begin{document}

\begin{abstract}
  This document contains a number of computations underlying some of
  the algorithms used in the code of QSoas. It serves both as a helper
  to write the code and a justification of how it works. See also the
  file \texttt{tests/doc/computations.tex} for the pendant of this
  file for the tests.
\end{abstract}

\section{Convolutions}

Here are the computations for the way \texttt{QSoas} convolves a
\emph{dataset} $f$ with a \emph{function} $g$. The convolution product
is given by the following integral:
%
\begin{equation}
  (f*g)(t) = \int_{t' = -\infty}^{\infty} f(t') g(t-t')\:\dd t'
\end{equation}
%
Assuming that the data points of $f$ are uniformly spread over $n$
intervals of length $\Delta t$ starting at $t_0$ (this means $n+1$
data points), this integral can be rewritten this way:
%
\begin{equation}
  (f*g)(t) = \sum_{i = 0}^{n-1}
  \int_{t_0 + i \Delta t}^{t_0 + (i+1) \Delta t}
  f(t') g(t-t')\:\dd t'
\end{equation}
%
The convolution process will produce a \emph{dataset} with the same
number and position of points as $f$. For all points $0 \leq i \leq n$, we have
therefore:
%
\begin{equation}
  (f*g)_i = (f*g)(t_0 + i\Delta t) = 
  \sum_{j = 0}^{n-1} \int_{t_0 + j \Delta t}^{t_0 + (j+1) \Delta t}
  f(t') g(t_0 + i\Delta t-t')\:\dd t'
\end{equation}

Lets look at the following integral defined for $0\leq j < n$ and $0
\leq i \leq n$:
%
\begin{equation}
  I_{i,j} = 
  \int_{t_0 + j \Delta t}^{t_0 + (j+1) \Delta t}
  f(t') g(t_0 + i\Delta t-t')\:\dd t'
\end{equation}
%
We will assume linear interpolation of the datapoints over each of the
intervals between data points, so that the following integral is in
fact:
%
\begin{equation}
  I_{i,j}  =
\int_{t_0 + j \Delta t}^{t_0 + (j+1) \Delta t}
\pab{f_j + \frac{t' - \pa{t_0 + j \Delta t}}{\Delta t} \pa{f_{j+1} - f_j}}
g(t_0 + i\Delta t-t')\:\dd t'
\end{equation}
%
Shifting the integral by $t_0 + j\Delta t$ yields:
%
\begin{align}
  I_{i,j}  &=
  \int_{0}^{ \Delta t}
  \pab{f_j + \frac{t'}{\Delta t} \pa{f_{j+1} - f_j}}
  g(\pab{i-j}\Delta t -t')\:\dd t'\\
  & = f_j \int_{0}^{\Delta t} g(\pab{i-j}\Delta t -t')\:\dd t'
  + \pa{f_{j+1} - f_j}
  \int_{0}^{\Deltat}
  \frac{t'}{\Delta t} g(\pab{i-j}\Delta t -t')\:\dd t'
\end{align}
%
Hence, the element $i$ of the convolution product can be written thus:
%
\begin{equation}
  (f*g)_i = \sum_{j = 0}^{n-1}
  \pab{
    A_{i-j}\, f_j + B_{i-j}\pa{f_{j+1} - f_j}
  }
  =
  \sum_{j = 0}^{n-1}
  \pab{
    \pa{A_{i-j}- B_{i-j}}f_j + B_{i-j}\,f_{j+1}
  }
\end{equation}
%
in which, for $-n < k \leq n$:
%
\begin{align}
  A_k &= \int_{0}^{\Delta t} g(k\Delta t -t')\:\dd t' =
  \int_{\pa{k-1} \Delta t}^{k \Delta t} g(t') \:\dd t'\\[4mm]
  B_k &= \int_{0}^{\Delta t}
  \frac{t'}{\Delta t} g(\pab{i-j}\Delta t -t')\:\dd t'
\end{align}
%
\obs{16.01.2024}{it may be better to parametrize from the end of
  the interval to cancel singularities on 0 for the first moment ?}

In any case, not all the integrals $A_k$ and $B_k$ have to be defined,
those who fall outside of the domain of definition of $g$ simply will
be assumed to be 0.

\paragraph{Computation of the integrals}
For now, let's just do an adaptive ``divide by two'' approach, it's
not a Gauss-Kronrod approach, but it will do (for now).




\end{document}
