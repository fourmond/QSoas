% -*- coding: utf-8; -*-
\documentclass{article}

%% This helper document is part of the test suite of QSoas
%% 
%% Copyright 2024 by CNRS/AMU

%% This program is free software; you can redistribute it and/or
%% modify it under the terms of the GNU General Public License as
%% published by the Free Software Foundation; either version 2 of the
%% License, or (at your option) any later version.

%% This program is distributed in the hope that it will be useful, but
%% WITHOUT ANY WARRANTY; without even the implied warranty of
%% MERCHANTABILITY or FITNESS FOR A PARTICULAR PURPOSE.  See the GNU
%% General Public License for more details.

%% You should have received a copy of the GNU General Public License
%% along with this program.  If not, see
%% <http://www.gnu.org/licenses/>.



\usepackage[english]{babel}
\usepackage[utf8]{inputenc}
\usepackage[T1]{fontenc}

\usepackage{amsmath}

\usepackage[svgnames]{xcolor}
\usepackage[colorlinks, urlcolor=blue, citecolor=DarkGreen,
linkcolor=blue]{hyperref}

\usepackage{graphicx}

\usepackage[a4paper,landscape=false,showframe=false,
margin=1.8cm]{geometry}

\usepackage[version=3]{mhchem}

\usepackage{siunitx}
\usepackage{booktabs}

\usepackage{mathpazo}
\usepackage[scaled]{helvet}
\usepackage{courier}

\usepackage{cite}
\usepackage{pifont}



%% A series of useful macros.
%%
%% These should be shared
\newcommand{\dd}{\mathrm{d}}
\newcommand{\ddt}[1]{\frac{\dd #1}{\dd t}}
\newcommand{\pa}[1]{\left(#1\right)}
\newcommand{\pab}[1]{\left[#1\right]}
\newcommand{\QSoas}{\texttt{QSoas}}


\begin{document}

\begin{abstract}
  This document contains a number of computations used in the more
  computational part of the tests. It serves both as a reminder of the
  rationale behind the tests and as a means to compute the results and
  verify them.
\end{abstract}

\section{Convolutions}

See tests in \texttt{.../...}.

The convolution product of two functions $f$ and $g$ is defined by the followingq
integral:
%
\begin{equation}
  (f\mathop{*}g)(t) = \int_{t' = -\infty}^{\infty} f(t') g(t-t') \dd t'
  \label{eq:convolution}
\end{equation}
%
This definition is of course subject to the definition domain of the
functions.

\subsection{Convolution of gaussians}

Let's look at the convolution product of two gaussians, with for
instance:
%
\begin{align}
  f(t) &= \exp \pa{-a\,t^2} &
  g(t) &= \exp \pa{-b\,t^2}
\end{align}
%
Then,
\begin{align}
  (f\mathop{*}g)(t) &=
  \int_{t' = -\infty}^{\infty}
  \exp \pa{-a\,t'^2}\exp \pa{-b\,[t-t']^2} \dd t'\\[3mm]
  &= 
  \int_{t' = -\infty}^{\infty}
  \exp \pab{-\pa{a+b}\,t'^2 + 2b\,t\,t' - b\,t^2} \dd t'\\[3mm]
  &= 
  \int_{t' = -\infty}^{\infty}
  \exp
  \pa{-\frac{1}{a+b}
    \pab{\pa{(a+b)t'}^2 - 2b\,t\,\pa{a+b}t' + \pa{b t}^2}
    - bt^2 + \frac{b^2t^2}{a+b}
  } \dd t'\\[3mm]
  &=
  \exp -\frac{a\,b\,t^2}{a+b} \times
  \int_{t' = -\infty}^{\infty}
  \exp
  \pa{-\frac{\pa{(a+b)t' - bt}^2}{a+b}}
  \dd t'\\[3mm]
  &= \sqrt{\frac{\pi}{a+b}} \times 
  \exp \pa{-\frac{a\,b\,t^2}{a+b}}
\end{align}


\section{Laplace transforms}

Let's consider the Laplace transform, defined by:
To solve these directly, it is useful to remember that the Laplace
transform of a derivative is simply given by:
%
\begin{equation}
  \tilde{f}(s) = \int_{t = 0}^{\infty} f(t) \exp \pa{-st} \dd t
  \label{eq:laplace:def}
\end{equation}
%
What is implemented in \QSoas{} is the reverse Laplace transform,
i.e. taking a function in the Laplace domain and turning back into the
time domain.

The Laplace transforms has many properties.
The Laplace transform of a derivative is given by:
%
\begin{equation}
  \int_{t = 0}^{\infty} \ddt{f}(t) \exp \pa{-st} \dd t
  = [f(t)\exp\pa{-st}]_{t = 0}^{t = \infty}
  + s \int_{t = 0}^{\infty} f(t) \exp \pa{-st}\dd t
  = s\tilde{f} - f(t = 0)
  \label{eq:laplace:ddt}
\end{equation}



\subsection{Simple kinetic models}

As a test for the reverse Laplace transform, let's consider the series
of irreversible reactions:
%
\begin{equation}
  \ce{C_{i} ->[k_i] C_{i+1}}
\end{equation}
%
in which the initial condition is $C_0 = 1$ and $C_{i>0} = 0$. The
differential equations governing the concentrations are:
%
\begin{align}
  \ddt{C_0} &= -k_0 C_0 &
  \ddt{C_i} &= k_{i-1} C_{i-1} - k_{i} C_{i} 
\end{align}
%
Taking the Laplace transform of the above equations (with the initial
conditions) and using equation~\eqref{eq:laplace:ddt} yields:
%
\begin{align}
  s \tilde{C}_0 - 1 &= -k_0 \tilde{C}_0 &
  s \tilde{C}_i = k_{i-1} \tilde{C}_{i-1} - k_{i} \tilde{C}_{i} 
\end{align}
%
In other words, the $\tilde{C}_i$ are defined by the following
recurrence:
%
\begin{align}
  \tilde{C}_0 &= \frac{1}{k_0 + s} &
  \tilde{C}_i &= \frac{k_{i-1} \tilde{C}_{i-1}}{k_i + s}
\end{align}
%
Thus:
\begin{align}
  \tilde{C}_0 &= \frac{1}{k_0 + s} & 
  \tilde{C}_1 &= \frac{k_{0}}{\pa{k_0 + s}\pa{k_1 + s}} &
  \tilde{C}_2 &= \frac{k_{0}\,k_1}{\pa{k_0 + s}\pa{k_1 + s}\pa{k_2 + s}}
\end{align}


\end{document}
